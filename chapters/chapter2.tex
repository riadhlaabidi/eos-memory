\chapter{Requirements gathering and specification}

\phantomsection
\section*{Introduction}
\addcontentsline{toc}{section}{Introduction}
In this chapter, we place significant focus on gathering requirements and transforming them 
into well-defined and documented specifications. This includes creating use cases and user stories, 
which provide valuable insights into system interactions and user workflows. Furthermore, we establish 
a prioritized product backlog, ensuring efficient resource allocation and timely delivery of the most 
critical and valuable features. 

\section{Requirements analysis}
By conducting a comprehensive analysis of the requirements, we aim to bridge the gap between 
the stakeholders' vision and the actual implementation of the product or system. 
This analysis helps us define clear and concise specifications that serve as the foundation for the design, 
development, and testing phases of the project.

\subsection{Identifying end-users}
Identifying end-users is a crucial step in requirements analysis as it helps determine the needs, 
expectations, and constraints of the target audience.

In our application, we were able to identify three types of actors:

\begin{itemize}
    \item \textbf{Admin}: Is responsible for overseeing user accounts, configuring and maintaining the 
    resources needed by agents or employees to ensure the smooth operation of the system inside the organization.
    
    \item \textbf{Agent}: Is an employee inside the organization in charge of executing actions on the 
    notification center: creating and scheduling notification campaigns including the creation of
    notification content, client base segmentation. 

    \item \textbf{Client}: Is a customer who is going to be targeted by notifications from the business 
    or organization they belong to. A customer should be able to receive notifications and set their 
    preferences for receiving notifications from that business. 
\end{itemize}

\phantomsection
\section*{Summary}
\addcontentsline{toc}{section}{Summary}
In this chapter, we translated requirements into specifications, encompassing the creation of use cases,
user stories, and a prioritized product backlog, ultimately setting the stage for subsequent phases 
of development.