\chapter{Sprint 0: Requirements gathering and Specification}

\phantomsection
\section*{Introduction}
\addcontentsline{toc}{section}{Introduction}
In this chapter, we place significant focus on gathering requirements and transforming them
into well-defined and documented specifications. This includes creating use cases and user stories,
which provide valuable insights into system interactions and user workflows. Furthermore, we establish
a prioritized product backlog, ensuring efficient resource allocation and timely delivery of the most
critical and valuable features.

\section{Requirements gathering}
By conducting a comprehensive analysis of the requirements, we aim to bridge the gap between
the stakeholders' vision and the actual implementation of the product or system.
This analysis helps us define clear and concise specifications that serve as the foundation for the design,
development, and testing phases of the project.

\subsection{Identifying end-users}
Identifying end-users is a crucial step in requirements analysis as it helps determine the needs,
expectations, and constraints of the target audience. In our application, we were able to identify
three types of actors:

\begin{itemize}
      \item \textbf{Administrator}: Is responsible for overseeing user accounts, configuring and maintaining the
            resources needed by agents or employees to ensure the smooth operation of the system inside the organization.

      \item \textbf{Agent}: Is an employee inside the organization in charge of executing actions on the
            notification center: creating and scheduling notification campaigns including the creation of
            notification content, client base segmentation.

      \item \textbf{Client}: Is a customer who is going to be targeted by notifications from the business
            or organization they belong to. A customer should be able to receive notifications and set their
            preferences for receiving notifications from that business.
\end{itemize}

\subsection{Functional requirements}
\label{freq}
Functional requirements define the specific actions, tasks, and behaviors that the product
must be able to perform in order to meet the needs of its end-users. These requirements
form the foundation of the system's functionality.

The functional requirements we captured for each actor in are outlined below.

\paragraph{Authentication and Profile settings}
\label{common-req}
\begin{itemize}
      \item \textbf{Sign in:} A registered user should be able to access the system by providing
            valid credentials.
      \item \textbf{Edit profile settings:} A logged in user should be able to edit his profile settings.
      \item \textbf{View statistics:} A logged in user should be able to view notification activity metrics
            on his dashboard.
\end{itemize}
\raggedbottom


\paragraph{Administrator requirements}
\label{admin-req}
\begin{itemize}
      \item \textbf{Manage agents:} An administrator should be able to add new agents, update,
            delete, desactivate accounts and reset passwords for existing agents.
      \item \textbf{Manage channels:} An administrator should be able to create new notification channels,
            update, delete, configure service providers for existing channels.
      \item \textbf{Manage topics:} An administrator should be able to create new notification topics,
            update, delete and configure topic's priority for existing ones.
\end{itemize}


\paragraph{Agent requirements}
\label{agent-req}
\begin{itemize}
      \item \textbf{Manage templates:} An agent should be able to create new notification templates, update and
            delete existing ones.
      \item \textbf{Manage triggers:} An agent should be able to create new notification triggers, configure
            the target audience the scheduling, update, change status and delete existing triggers.
      \item \textbf{Manage audiences:} An agent should be able create new segments of users based on a
            criteria, update and delete existing ones.
      \item \textbf{View logs:} An agent should be able to view logs of sent notifications and their statuses.
\end{itemize}

\paragraph{Client requirements}
\label{client-req}
\begin{itemize}
      \item \textbf{Manage notification preference:} A client should be able to edit his notification preferences,
            channels and frequency of receiving notifications.
      \item \textbf{View notification history:} A client should be able to checkout a history of his received
            notifications (for in-app notifications).
\end{itemize}

\subsection{Non-Functional requirements}
\label{nfr}
When designing a notification system, various technical requirements need to be considered to ensure
its effectiveness, reliability, and scalability. Here are the key requirements that should be addressed:

\begin{itemize}
      \item \textbf{Security:} The system shall enforce secure communication protocols, such as \acrshort{https},
            to protect sensitive data during transmissionn, also data and preferences stored in the system shall be securely
            encrypted to prevent unauthorized access or data breaches.
      \item \textbf{Real-time:} The system shall deliver notifications in real-time or near real-time
            to ensure timely communication, messages and notifications should be delivered with minimal delay for high
            priority topics.
      \item \textbf{Scalability:} The system should be designed to handle a high volume of concurrent users
            and notifications without compromising performance and the system architecture should be scalable, allowing for
            horizontal scaling by adding more servers or utilizing cloud-based infrastructure as the user base grows.
      \item \textbf{Customizability:} The system should provide flexibility and customizability to meet
            the specific branding and user experience requirements of different organizations. Also the system should
            allow customization of user preferences to provide a personalized experience.
\end{itemize}

\section{Scrum implementation overview}
\label{spec}
This section provides an overview of the Scrum implementation in our project It includes a global use case diagram, showcasing the system's
high-level functionalities, followed by the presentation of the product backlog and the planning
of the sprints.

\subsection{Global use case diagram}
Using \acrshort{uml} use case diagrams to model the requirements allows for a visual representation
of the interactions between actors and the system, providing a clear and concise way to specify
the functionalities and behaviors expected from the product.

The figure \ref{g-usecase} illustrates the global use case diagram we modeled for our notification system: \\

\begin{figure}[hbt!]
      \centering
      \includesvg[]{diagrams/usecase/global}
      \caption{Notification Center global use case diagram}
      \label{g-usecase}
\end{figure}

\subsection{Product backlog}
The Product Backlog serves as a dynamic and living artifact that captures and organizes the
ever-evolving list of features, functionalities, and enha   ncements desired for a software product.

\begin{itemize}
      \item \textbf{Epic:} Is a large body of work that can be broken down into a number of smaller stories.
      \item \textbf{User story:} Is an informal, general explanation of a software feature written from the
            perspective of the end user or customer, in the form of: \\ “As a [persona], I [want to], [so that].” \\
            % \item \textbf{Estimation:} An estimate of the overall effort required to fully implement a product
            %       backlog item or any other piece of work, in our case we used the number of working hours. \\
\end{itemize}

\begin{longtable}{ | m{0.18\textwidth}  | m{0.77\textwidth} | }
      \hline
      \textbf{Epic}                                           & \textbf{User story}                                                                                                                                                   \\
      \hline
      \endfirsthead
      \hline
      \textbf{Epic}                                           & \textbf{User story}                                                                                                                                                   \\
      \hline
      \endhead
      \hline
      \endfoot
      \endlastfoot
      \multirow[t]{2}{5em}{Authentication}                    & As a new user, I want to be able to create an account so that I can use the notification center.                                                                      \\
      \cline{2-2}
                                                              & As a registered user, I want to be able to log into my account securely using my email and password.                                                                  \\
      \hline
      \multirow[t]{4}{5em}{Agents management}                 & As an administrator, I want to be able to create agents so that I can add them to the notification center.                                                            \\
      \cline{2-2}
                                                              & As an administrator, I want to be able to list agents so that I can view all registered agents.                                                                       \\
      \cline{2-2}
                                                              & As an administrator, I want to be able to edit agents so that I can modify their information.                                                                         \\
      \cline{2-2}
                                                              & As an administrator, I want to be able to delete agents so that I can get rid of no longer needed agents.                                                             \\
      \hline
      \multirow[t]{4}{5em}{Users management}                  & As an administrator/agent, I want to be able to create users so that I can send them notifications.                                                                   \\
      \cline{2-2}
                                                              & As an administrator/agent, I want to be able to list users so that I can view all created users.                                                                      \\
      \cline{2-2}
                                                              & As an administrator/agent, I want to be able to edit users so that I can modify their information.                                                                    \\
      \cline{2-2}
                                                              & As an administrator/agent, I want to be able to delete users so that I can get rid of no longer needed users.                                                         \\
      \hline
      \multirow[t]{4}{5em}{Audience management}               & As an administrator/agent, I want to be able to create an audience so that I can target specific individuals based on a criteria.                                     \\
      \cline{2-2}
                                                              & As an administrator/agent, I want to be able to list audiences so that I can view all created segments.                                                               \\
      \cline{2-2}
                                                              & As an administrator/agent, I want to be able to edit an audience so that I can change selection criteria and segment configurations.                                  \\
      \cline{2-2}
                                                              & As an administrator/agent, I want to be able to delete a group of users so that I can get rid of no longer needed groups..                                            \\
      \hline
      \multirow[t]{4}{5em}{Notification channels management}  & As an administrator/agent, I want to be able to create notification channels so that I can send notifications through these channels.                                 \\
      \cline{2-2}
                                                              & As an administrator/agent, I want to be able to list notification channels so that I can view all created channels.                                                   \\
      \cline{2-2}
                                                              & As an administrator, I want to be able to edit notification channels so that I can modify or update their configurations.                                             \\
      \cline{2-2}
                                                              & As an administrator/agent, I want to be able to delete notification channels so that I can get rid of no longer used channels.                                        \\
      \hline
      \multirow[t]{4}{5em}{Notification templates management} & As an agent, I want to be able to add notification templates so that I can send notifications based on that template.                                                 \\
      \cline{2-2}
                                                              & As an agent, I want to be able to list notification templates so that I can view all created templates.                                                               \\
      \cline{2-2}
                                                              & As an agent, I want to be able to edit notification templates so that I can keep them up to date.                                                                     \\
      \cline{2-2}
                                                              & As an agent, I want to be able to delete templates so that I can get rid of no longer used templates.                                                                 \\
      \hline
      Notification preferences management                     & As a user, I want to be able to set my notification preferences, so that I can receive notifications from the channels I want.                                        \\
      \hline
      \multirow[t]{4}{5em}{Notification triggers management}  & As an agent, I want to be able to create triggers for notifications so that I can schedule notifications to be sent automatically.                                    \\
      \cline{2-2}
                                                              & As an agent, I want to be able to list triggers for notifications so that I can view all created triggers.                                                            \\
      \cline{2-2}
                                                              & As an agent, I want to be able to edit notifications triggers so that I can modify or update its configurations.                                                      \\
      \cline{2-2}
                                                              & As an agent, I want to be able to delete notification triggers so that I can get rid of outdated and no longer used triggers.                                         \\
      \hline
      \multirow[t]{4}{5em}{Notification history}              & As a user, I want to be able to list notification history so that I can review my received notifications whenever I want.                                             \\
      \cline{2-2}
                                                              & As an administrator/agent, I want to be able to list sent notification logs so that I can review all sent notifications.                                              \\
      \hline
      Dashboard                                               & As an administrator/agent I want to be able to view metrics on my dashboard so that I can get an overview on important statistics related to notification activities. \\
      \hline
      \caption{Product backlog}
\end{longtable}

\subsection{Releases and sprints planning}
A release refers to the deployment of a specific version or update of a software product. It involves meticulous planning
and coordination to ensure successful delivery.

In the following table, we present the detailed planning of the releases, outlining the sprints included in each release,
and the epics included in each sprint. \\

\begin{table}[hbt!]
      \begin{tabular}{ | m{0.25\textwidth} | m{0.25\textwidth} | m{0.42\textwidth} | }
            \hline
            \textbf{Release}             & \textbf{Sprints}            & \textbf{Epics}                      \\
            \hline
            \multirow{2}{5em}{Release 1}
                                         & \multirow{3}{5em}{Sprint 1} & Authentication                      \\
            \cline{3-3}
                                         &                             & Agents management                   \\
            \cline{3-3}
                                         &                             & Users management                    \\
            \cline{2-3}

                                         & \multirow{2}{5em}{Sprint 2} & Audiences management                \\
            \cline{3-3}
                                         &                             & Notification channels management    \\
            \hline
            \multirow{2}{5em}{Release 2} & \multirow{2}{5em}{Sprint 3} & Notification templates management   \\
            \cline{3-3}
                                         &                             & Notification preferences management \\
            \cline{2-3}
                                         & \multirow{3}{5em}{Sprint 4} & Notification triggers management    \\
            \cline{3-3}
                                         &                             & Notification history                \\
            \cline{3-3}
                                         &                             & Dashboard                           \\
            \hline
      \end{tabular}
      \caption{Releases and sprints planning}
      \label{tab:planning}
\end{table}

\section{Development infrastructure}
In this section, we will discuss the development infrastructure established for the successful execution
of the project. The primary focus will be on the software architecture. Additionally, we will elaborate on the curated set of tools and frameworks that were employed to support
and streamline the development workflow.


\subsection{Software architecture}
To implement our notification center, we will pick a client-server architecture as we need control
over the service. This architecture is the fundamental building block of the web and works on a
request-response model. The client sends the request to the server for information and the server
responds with it.

% TODO: Elaborate & Illustration for client server architecture with 3-tiers

We agreed to adopt a monolithic approach, where all the application modules and components are tightly
integrated into a single codebase. This decision was based on several factors such as the project's
scope, timeline, and the team's familiarity with monolithic architectures. Furthermore, for the envisioned
requirements, a monolithic design offered simplicity and ease of development, deployment, and maintenance.

% TODO: Illustraion if needed

\pagebreak
\phantomsection
\section*{Summary}
\addcontentsline{toc}{section}{Summary}
In this chapter we laid the groundwork for the project by comprehensively gathering and analyzing
the requirements. By identifying end-users, defining functional and non-functional requirements,
and creating a clear specification through use case diagrams, elaborating the product backlog,
and planning sprints, we have set the stage for the subsequent phases of development.

The insights gained from this chapter will prove invaluable in ensuring the successful implementation
of the project and the satisfaction of the end-users. In the following chapter, we will start executing
our project providing the practical realization of the initial release.
